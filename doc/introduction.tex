\chapter{Introduction}

The GNU Hurd is a multi-server operating system running on top of a
microkernel (currently Mach variants).  The core motivation of the
Hurd is the following:

\begin{quote}
  \emph{The operating system should enable its users to share the
    resources of the system without harming each other.}
\end{quote}

The focus is on the user, the system should try to allow the user to
do anything that is not harmful for other users.  Many operating
systems either restrict what the user can do to be more secure, while
others allow the user to do everything, but fail on protecting the
users from each other effectively.

The Hurd is designed to minimize the system code that the user is
required to use, while allowing the user to use, ignore or replace the
remaining system code, and this without harming other users.

So while the L4 microkernel tries to minimize the policy that the
kernel enforces on the software running on it, the Hurd tries to
minimize the policy that the operating system enforces on its users.
Furthermore, the Hurd also aims to provide a POSIX compatible general
purpose operating system.  However, this POSIX personality of the Hurd
is provided for convenience only, and to make the Hurd useful.  Other
personalities can be implemented and used by the users of the system
along with the POSIX personality.  This default personality of the
Hurd also provides some convenient features that allow the user to
extend the system so that all POSIX compatible programs can take
advantage of it.

These notes are a moving target in the effort to find the best
strategy to port the Hurd to the L4 microkernel.

\begin{comment}
  Remarks about the history of a certain feature and implementation
  details are set in a smaller font and separated from the main text,
  just like this paragraph.  Because this is work in progress, there
  are naturally a lot of such comments.
\end{comment}


