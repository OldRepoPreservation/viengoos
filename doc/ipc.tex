\chapter{Inter-process communication (IPC)}
\label{ipc}

The Hurd requires a capability system.  Capabilities are used to proof
your identity to other servers (authentication), and access
server-side implemented objects like devices, files, directories,
terminals, and other things.  The server can use a capability for
whatever it wants.  Capabilities provide interfaces.  Interfaces can
be invoked by sending messages to the capability.  In L4, this means
that a message is sent to a thread in the server providing the
capability, with the identifier for the capability in the message.

Capabilities are protected objects.  Access to a capability needs to
be granted by the server.  Once you have a capability, you can copy it
to other tasks (if the server permits it, which is usually the case).
In the Hurd, access to capabilities is always granted to a whole task,
not to individual threads.

\begin{comment}
  There is no reason for the server not to permit it, because the
  holder of the capability could also just act as a proxy for the
  intended receiver instead copying the capability to it.  The
  operation might fail anyway, for example because of resource
  shortage, in particular if the server puts a quota on the number of
  capabilities a user can hold.
\end{comment}

Capabilities provide two essential services to the Hurd.  They are
used to restrict access to a server function, and they are the
standard interface the components in the Hurd use to communicate with
each others.  Thus, it is important that their implementation is fast
and secure.

\begin{comment}
  There are several ways to implement such a capability system.  A
  more traditional design would be a global, trusted capability server
  that provides capabilities to all its users.  The L4 redirector
  could be used to reroute all client traffic automatically through
  this server.  This approach has several disadvantages:

  \begin{itemize}
  \item It adds a lot of overhead to every single RPC, because all
    traffic has to be routed through the capability server, which must
    then perform the authentication on the server's behalf.
  \item It would be difficult to copy a capability to another task.
    Either the cap server would have to provide interfaces for clients
    to do it, or it would be have to know the message format for every
    interface and do it automatically.
  \item It would be a single point of failure.  If it had a bug and
    crashed, the whole system would be affected.
  \item Users could not avoid it, it would be enforced system code.
  \item It is inflexible.  It would be hard to replace or extend at
    run-time.
  \end{itemize}
  
  Another approach is taken by CORBA with IORs.  IORs contain long
  random numbers which allow the server to identify a user of an
  object.  This approach is not feasible for the following reasons:

  \begin{itemize}
  \item Even good random numbers can be guessed.  Long enough random
    numbers can reduce the likelihood to arbitrary small numbers,
    though (below the probability of a hardware failure).
  \item Good random numbers are in short supply, and is slow to
    generate.  Good pseudo random is faster, but it is still difficult
    to generate.  The random number generator would become a critical
    part of the operating system.
  \item The random number had to be transfered in every single
    message.  Because it would have to be long, it would have a
    significant negative impact on IPC performance.
  \end{itemize}
\end{comment}

The Hurd implements the capability system locally in each task.  A
common default implementation will be shared by all programs.
However, a malicious untrusted program can do nothing to disturb the
communication of other tasks.  A capability is identified in the
client task by the server thread and a local identifier (which can be
different from client to client).  The server thread will receive
messages for the capabilities.  The first argument in the message is
the capability identifier.  Although every task can get different IDs
for the same capability, a well-behaving server will give the same ID
to a client which already has a capability and gets the same
capability from another client.  So clients can compare capability IDs
from the server numerically to check if two capabilities are the same,
but only if one of the two IDs is received while the client already
had the other one.

Because access to a capability must be restricted, the server needs to
be careful in only allowing registered and known users to access the
capability.  For this, the server must be sure that it can determine
the sender of a message.  In L4, this is easy on the surface: The
kernel provides the receiving thread with the sender's thread ID,
which also contains the task ID in the version field.  However, the
server must also know for sure if this task is the same task that it
gave access to the capability.  Comparing the task IDs numerically is
not good enough, the server must also somehow have knowledge or
influence on how task IDs are reused when tasks die and are created.

The same is true for the client, of course, which trusts the server
and thus must be sure that it is not tricked into trusting on
unreliable data from an imposter, or sends sensitive data to it.

\begin{comment}
  The \texttt{task} server wants to reuse thread numbers because that
  makes best use of kernel memory.  Reusing task IDs, the version
  field of a thread ID, is not so important, but there are only 14
  bits for the version field (and the lower six bits must not be all
  zero).  So a thread ID is bound to be reused eventually.
  
  Using the version field in a thread ID as a generation number is not
  good enough, because it is so small.  Even on 64-bit architectures,
  where it is 32 bit long, it can eventually overflow.
\end{comment}

The best way to prevent that a task can be tricked into talking to an
imposter is to have the \texttt{task} server notify the task if the
communication partner dies.  The \texttt{task} server must guarantee
that the task ID is not reused until all tasks that got such a
notification acknowledge that it is processed, and thus no danger of
confusion exists anymore.

The \texttt{task} server provides references to task IDs in form of
\emph{task info capabilities}.  If a task has a task info capability
for another task, it prevents that this other task's task ID is reused
even if that task dies, and it also makes sure that task death
notifications are delivered in that case.

\begin{comment}
  Because only the \texttt{task} server can create and destroy tasks,
  and assign task IDs, there is no need to hold such task info
  capabilities for the \texttt{task} server, nor does the
  \texttt{task} server need to hold task info capabilities for its
  clients.  This avoids the obvious bootstrap problem in providing
  capabilities in the \texttt{task} server.  This will even work if
  the \texttt{task} server is not the real \texttt{task} server, but a
  proxy task server (see section \ref{proxytaskserver} on page
  \pageref{proxytaskserver}).
\end{comment}

As task IDs are a global resource, care has to be taken that this
approach does not allow for a DoS-attack by exhausting the task ID
number space, see section \ref{taskinfocap} on page
\pageref{taskinfocap} for more details.


\section{Capabilities}

This subsection contains implementation details about capabilities.

A server will usually operate on objects, and not capabilities.  In
the case of a filesystem, this could be file objects, for example.

\begin{comment}
  In the Hurd, filesystem servers have to keep different objects for
  each time a file is looked up (or ``opened''), because some state,
  for example authentication, open flags and record locks, are
  associated not with the file directly, but with this instance of
  opening the file.  Such a state structure (``credential'') will also
  contain a pointer and reference to the actual file node.  For
  simplicity, we will assume that the capability is associated with a
  file node directly.
\end{comment}

To provide access to the object to another task, the server creates a
capability, and associates it with the object (by setting a hook
variable in the capability).  From this capability, the server can
either create send references to itself, or to other tasks.  If the
server creates send references for itself, it can use the capability
just as it can use capabilities implemented by other servers.  This
makes access to locally and remotely implemented capabilities
identical.  If you write code to work on capabilities, it can be used
for remote objects as well as for local objects.

If the server creates a send reference for another task (a client), a
new capability ID will be created for this task.  This ID will only be
valid for this task, and should be returned to the client.

The client itself will create a capability object from this capability
ID.  The capability will also contain information about the server,
for example the server thread which should be used for sending
messages to the capability.

If the client wants to send a message, it will send it to the provided
server thread, and use the capability ID it got from the server as the
first argument in the RPC.  The server receives the message, and now
has to look up the capability ID in the list of capabilties for this
task.

\begin{comment}
  The server knows the task ID from the version field of the sender's
  thread ID.  It can look up the list of capabilities for this task in
  a hash table.  The capability ID can be an index into an array, so
  the server only needs to perform a range check.  This allows to
  verify quickly that the user is allowed to access the object.
  
  This is not enough if several systems run in parallel on the same
  host.  Then the version ID for the threads in the other systems will
  not be under the control of the Hurd's \texttt{task} server, and can
  thus not be trusted.  The server can still use the version field to
  find out the task ID, which will be correct \emph{if the thread is
    part of the same subsystem}.  It also has to verify that the
  thread belongs to this subsystem.  Hopefully the subsystem will be
  encoded in the thread ID.  Otherwise, the \texttt{task} server has
  to be consulted (and, assuming that thread numbers are not shared by
  the different systems, the result can be cached).
\end{comment}

The server reads out the capability associated with the capability ID,
and invokes the server stub according to the message ID field in the
message.

After the message is processed, the server sends it reply to the
sender thread with a zero timeout.

\begin{comment}
  Servers must never block on sending messages to clients.  Even a
  small timeout can be used for DoS-attacks.  The client can always
  make sure that it receives the reply by using a combined send and
  receive operation together with an infinite timeout.
\end{comment}

The above scheme assumes that the server and the client already have
task info caps for the respective other task.  This is the normal
case, because acquiring these task info caps is part of the protocol
that is used when a capability is copied from one task to another.


\subsection{Bootstrapping a client-server connection}
\label{ipcbootstrap}

If the client and the server do not know about each other yet, then
they can bootstrap a connection without support from any other task
except the \texttt{task} server.  The purpose of the initial handshake
is to give both participants a chance to acquire a task info cap for
the other participants task ID, so they can be sure that from there on
they will always talk to the same task as they talked to before.

\subsubsection{Preconditions}
The client knows the thread ID of the server thread that receives and
processes the bootstrap messages.  Some other task might hold a task
info capability to the server the client wants to connect to.

\begin{comment}
  If no such other tasks exists, the protocol will still work.
  However, the client might not get a connection to the server that
  run at the time the client started the protocol, but rather to the
  server that run at the time the client acquired the task info cap
  for the server's task ID (after step 1 below).
 
  This is similar to how sending signals works in Unix: Technically,
  at the time you write \texttt{kill 203}, and press enter, you do not
  know if the process with the PID 203 you thought of will receive the
  signal, or some other process that got the PID in the time between
  you getting the information about the PID and writing the
  \texttt{kill}-command.
\end{comment}

FIXME: Here should be the pseudo code for the protocol.  For now, you
have to take it out of the long version.

\begin{enumerate}
  
\item The client acquires a task info capability for the server's task
  ID, either directly from the \texttt{task} server, or from another
  task in a capability copy.  From that point on, the client can be
  sure to always talk to the same task when talking to the server.
  
  Of course, if the client already has a task info cap for the server
  it does not need to do anything in this step.

\begin{comment}
  As explained above, if the client does not have any other task
  holding the task info cap already, it has no secure information
  about what this task is for which it got a task info cap.
\end{comment}

\item The client sends a message to the server, requesting the initial
  handshake.
  
\item The server receives the message, and acquires a task info cap
  for the client task (directly from the \texttt{task} server).
  
  Of course, if the server already has a task info cap for the client
  it does not need to do anything in this step.

\begin{comment}
  At this point, the server knows that future messages from this task
  will come from the same task as it got the task info cap for.
  However, it does not know that this is the same task that sent the
  initial handshake request in step 2 above.  This shows that there is
  no sense in verifying the task ID or perform any other
  authentication before acquiring the task info cap.
\end{comment}

\item The server replies to the initial handshake request with an
  empty reply message.

\begin{comment}
  Because the reply now can go to a different task than the request
  came from, sending the reply might fail.  It might also succeed and
  be accepted by the task that replaced the requestor.  Or it might
  succeed normally.  The important thing is that it does not matter to
  the server at all.  It would have provided the same ``service'' to
  the ``imposter'' of the client, if he had bothered to do the
  request.  As no authentication is done yet, there is no point for
  the server to bother.
  
  This means however, that the server needs to be careful in not
  consuming too many resources for this service.  However, this is
  easy to achieve.  Only one task info cap per client task will ever
  be held in the server.  The server can either keep it around until
  the task dies (and a task death notification is received), or it can
  clean it up after some timeout if the client does not follow up and
  do some real authentication.
\end{comment}

\item The client receives the reply message to its initial handshake
  request.
  
\item The client sends a request to create its initial capability.
  How this request looks depends on the type of the server and the
  initial capabilities it provides.  Here are some examples:

  \begin{itemize}
  \item A filesystem might provide an unauthenticated root directory
    object in return of the underlying node capability, which is
    provided by the parent filesystem and proves to the filesystem
    that the user was allowed to look up the root node of this
    filesystem (see section \ref{xfslookup} on page
    \pageref{xfslookup}).

    \begin{comment}
      In this example, the parent filesystem will either provide the
      task info cap for the child filesystem to the user, or it will
      hold the task info cap while the user is creating their own
      (which the user has to verify by repeating the lookup, though).
      Again, see section \ref{xfslookup} on page \pageref{xfslookup}.
      
      The unauthenticated root directory object will then have the be
      authenticated using the normal reauthentication mechanism (see
      section \ref{auth} on pageref{auth}).  This can also be combined
      in a single RPC.
    \end{comment}
    
  \item Every process acts as a server that implements the signal
    capability for this process.  Tasks who want to send a signal to
    another task can perform the above handshake, and then provide
    some type of authentication capability that indicates that they
    are allowed to send a signal.  Different authentication
    capabilities can be accepted by the signalled task for different
    types of signals.

    \begin{comment}
      The Hurd used to store the signal capability in the proc server,
      where authorized tasks could look it up.  This is no longer
      possible because a server can not accept capabilities
      implemented by untrusted tasks, see below.
    \end{comment}
  \end{itemize}
  
\item The server replies with whatever capability the client
  requested, provided that the client could provide the necessary
  authentication capabilities, if any.

  \begin{comment}
    It is not required that the server performs any authentication at
    all, but it is recommended, and all Hurd servers will do so.
    
    In particular, the server should normally only allow access from
    tasks running in the same system, if running multiple systems on
    the same host is possible.
  \end{comment}
\end{enumerate}

\subsubsection{Result}
The client has a task info capability for the server and an
authenticated capability.  The server has a task info capability for
the client and seen some sort of authentication for the capability it
gave to the client.

\begin{comment}
  If you think that the above protocol is complex, you have seen
  nothing yet!  Read on.
\end{comment}


\subsection{Returning a capability from a server to a client}

Before we go on to the more complex case of copying a capability from
one client to another, let us point out that once a client has a
capability from a server, it is easy for the server to return more
capabilities it implements to the client.

The server just needs to create the capability, acquire a capability
ID in the client's cap ID space, and return the information in the
reply RPC.

FIXME: Here should be the pseudo code for the protocol.  For now, you
have to take it out of the long version.

\begin{comment}
  The main point of this section is to point out that only one task
  info capability is required to protect all capabilities provided to
  a single task.  The protocols described here always assume that no
  task info caps are held by anyone (except those mentioned in the
  preconditions).  In reality, sometimes the required task info caps
  will already be held.
\end{comment}


\subsection{Copying a capability from one client to another task}

The most complex operation in managing capabilities is to copy or move
a capability from the client to another task, which subsequently
becomes a client of the server providing the capability.  The
difficulty here lies in the fact that the protocol should be fast, but
also robust and secure.  If any of the participants dies unexpectedly,
or any of the untrusted participants is malicious, the others should
not be harmed.

\subsubsection{Preconditions}
The client $C$ has a capability from server $S$ (this implies that $C$
has a task info cap for $S$ and $S$ has a task info cap for $C$).  It
wants to copy the capability to the destination task $D$.  For this,
it will have to make RPCs to $D$, so $C$ has also a capability from
$D$ (this implies that $C$ has a task info cap for $D$ and $D$ has a
task info cap for $C$).  Of course, the client $C$ trusts its servers
$S$ and $D$.  $D$ might trust $S$ or not, and thus accept or reject
the capability that $C$ wants to give to $D$.  $S$ does not trust
either $C$ or $D$.
  
The \texttt{task} server is also involved, because it provides the
task info capabilities.  Everyone trusts the \texttt{task} server they
use.  This does not need to be the same one for every participant.

FIXME: Here should be the pseudo code for the protocol.  For now, you
have to take it out of the long version.

\begin{enumerate}
\item The client invokes the \verb/cap_ref_cont_create/ RPC on the
  capability, providing the task ID of the intended receiver $D$ of
  the capability.
  
\item The server receives the \verb/cap_ref_cont_create/ RPC from the
  client.  It requests a task info cap for $D$ from its trusted task
  server, under the constraint that $C$ is still living.

  \begin{comment}
    A task can provide a constraint when creating a task info cap in
    the \texttt{task} server.  The constraint is a task ID.  The task
    server will only create the task info cap and return it if the
    task with the constraint task ID is not destroyed.  This allows
    for a task requesting a task info capability to make sure that
    another task, which also holds this task info cap, is not
    destroyed.  This is important, because if a task is destroyed, all
    the task info caps it held are released.

    In this case, the server relies on the client to hold a task info
    cap for $D$ until it established its own.  See below for what can
    go wrong if the server would not provide a constraint and both,
    the client and the destination task would die unexpectedly.
  \end{comment}
  
  Now that the server established its own task info cap for $D$, it
  creates a reference container for $D$, that has the following
  properties:

  \begin{itemize}
  \item The reference container has a single new reference for the
    capability.
    
  \item The reference container has an ID that is unique among all
    reference container IDs for the client $C$.
    
  \item The reference container is associated with the client $C$.  If
    $C$ dies, and the server processes the task death notification for
    it, the server will destroy the reference container and release
    the capability reference it has (if any).  All resources
    associated with the reference container will be released.  If this
    reference container was the only reason for $S$ to hold the task
    info cap for $D$, the server will also release the task info cap
    for $D$.
    
  \item The reference container is also associated with the
    destination task $D$.  If $D$ dies, and the server processes the
    task death notification for it, the server will release the
    capability reference that is in the reference container (if any).
    It will not destroy the part of the container that is associated
    with $C$.
  \end{itemize}

  The server returns the reference container ID $R$ to the client.

\item The client receives the reference container ID $R$.

  \begin{comment}
    If several capabilities have to be copied in one message, the
    above steps need to be repeated for each capability.  With
    appropriate interfaces, capabilities could be collected so that
    only one call per server has to be made.  We are assuming here
    that only one capability is copied.
  \end{comment}

\item The client sends the server thread ID $T$ and the reference
  container ID $R$ to the destination task $D$.
  
\item The destination task $D$ receives the server thread ID $T$ and
  the reference container ID $R$ from $C$.
  
  It now inspects the server thread ID $T$, and in particular the task
  ID component of it.  $D$ has to make the decision if it trusts this
  task to be a server for it, or if it does not trust this task.
  
  If $D$ trusts $C$, it might decide to always trust $T$, too,
  irregardless of what task contains $T$.
  
  If $D$ does not trust $C$, it might be more picky about the task
  that contains $T$.  This is because $D$ will have to become a client
  of $T$, so it will trust it.  For example, it will block on messages
  it sends to $T$.

  \begin{comment}
    If $D$ is a server, it will usually only accept capabilities from
    its client that are provided by specific other servers it trusts.
    This can be the authentication server, for example (see section
    \ref{auth} on page \pageref{auth}).
    
    Usually, the type of capability that $D$ wants to accept from $C$
    is then further restricted, and only one possible trusted server
    implements that type of capabilities.  Thus, $D$ can simply
    compare the task ID of $T$ with the task ID of its trusted server
    (authentication server, ...) to make the decision if it wants to
    accept the capability or not.
  \end{comment}
  
  If $D$ does not trust $T$, it replies to $C$ (probably with an error
  value indicating why the capability was not accepted).  In that
  case, jump to step \ref{copycapout}.
  
  Otherwise, it requests a task info cap for $S$ from its trusted task
  server, under the constraint that $C$ is still living.
  
  Then $D$ sends a \verb/cap_ref_cont_accept/ RPC to the server $S$,
  providing the task ID of the client $C$ and the reference container
  ID $R$.

\begin{comment}
  \verb/cap_ref_cont_accept/ is one of the few interfaces that is not
  sent to a (real) capability, of course.  Nevertheless, it is part of
  the capability object interface, hence the name.  You can think of
  it as a static member in the capability class, that does not require
  an instance of the class.
\end{comment}
  
\item The server receives the \verb/cap_ref_cont_accept/ RPC from the
  destination task $D$.  It verifies that a reference container exists
  with the ID $R$, that is associated with $D$ and $C$.
  
  \begin{comment}
    The server will store the reference container in data structures
    associated with $C$, under an ID that is unique but local to $C$.
    So $D$ needs to provide both information, the task ID and the
    reference container ID of $C$.
  \end{comment}

  If that is the case, it takes the reference from the reference
  container, and creates a capability ID for $D$ from it.  The
  capability ID for $D$ is returned in the reply message.
  
  From that moment on, the reference container is deassociated from
  $D$.  It is still associated with $C$, but it does not contain any
  reference for the capability.

  \begin{comment}
    It is not deassociated from $C$ and removed completely, so that
    its ID $R$ (or at least the part of it that is used for $C$) is
    not reused.  $C$ must explicitely destroy the reference container
    anyway because $D$ might die unexpectedly or return an error that
    gives no indication if it accepted the reference or not.
  \end{comment}
  
\item The destination task $D$ receives the capability ID and enters
  it into its capability system.  It sends a reply message to $C$.

  \begin{comment}
    If the only purpose of the RPC was to copy the capability, the
    reply message can be empty.  Usually, capabilities will be
    transfered as part of a larger operation, though, and more work
    will be done by $D$ before returning to $C$.
  \end{comment}
  
\item \label{copycapout} The client $C$ receives the reply from $D$.
  Irregardless if it indicated failure or success, it will now send
  the \verb/cap_ref_cont_destroy/ message to the server $S$, providing
  the reference container $R$.

  \begin{comment}
    This message can be a simple message.  It does not require a reply
    from the server.
  \end{comment}
  
\item The server receives the \verb/cap_ref_cont_destroy/ message and
  removes the reference container $R$.  The reference container is
  deassociated from $C$ and $D$.  If this was the only reason that $S$
  held a task info cap for $D$, this task info cap is also released.

  \begin{comment}
    Because the reference container can not be deassociated from $C$
    by any other means than this interface, the client does not need
    to provide $D$.  $R$ can not be reused without the client $C$
    having it destroyed first.  This is different from the
    \verb/cap_ref_cont_accept/ call made by $D$, see above.
  \end{comment}

\end{enumerate}

\subsubsection{Result}
For the client $C$, nothing has changed.  The destination task $D$
either did not accept the capability, and nothing has changed for it,
and also not for the server $S$.  Or $D$ accepted the capability, and
it now has a task info cap for $S$ and a reference to the capability
provided by $S$.  In this case, the server $S$ has a task info cap for
$D$ and provides a capability ID for this task.

The above protocol is for copying a capability from $C$ to $D$.  If
the goal was to move the capability, then $C$ can now release its
reference to it.

\begin{comment}
  Originally we considered to move capabilities by default, and
  require the client to acquire an additional reference if it wanted
  to copy it instead.  However, it turned out that for the
  implementation, copying is easier to handle.  One reason is that the
  client usually will use local reference counting for the
  capabilities it holds, and with local reference counting, one
  server-side reference is shared by many local references.  In that
  case, you would need to acquire a new server-side reference even if
  you want to move the capability.  The other reason is cancellation.
  If an RPC is cancelled, and you want to back out of it, you need to
  restore the original situation.  And that is easier if you do not
  change the original situation in the first place until the natural
  ``point of no return''.
\end{comment}

The above protocol quite obviously achieves the result as described in
the above concluding paragraph.  However, many other, and often
simpler, protocols would also do that.  The other protocols we looked
at are not secure or robust though, or require more operations.  To
date we think that the above is the shortest (in particular in number
of IPC operations) protocol that is also secure and robust (and if it
is not we think it can be fixed to be secure and robust with minimal
changes).  We have no proof for its correctness.  Our confidence comes
from the scrutiny we applied to it.  If you find a problem with the
above protocol, or if you can prove various aspects of it, we would
like to hear about it.

To understand why the protocol is laid out as it is, and why it is a
secure and robust protocol, one has to understand what could possibly
go wrong and why it does not cause any problems for any participant if
it follows its part of the protocol (independent on what the other
participants do).  In the following paragraphs, various scenarios are
suggested where things do not go as expected in the above protocol.
This is probably not a complete list, but it should come close to it.
If you find any other problematic scenario, again, let us know.

\begin{comment}
  Although some comments like this appear in the protocol description
  above, many comments have been spared for the following analysis of
  potential problems.  Read the analysis carefully, as it provides
  important information about how, and more importantly, why it works.
\end{comment}

\subsubsection{The server $S$ dies}
What happens if the server $S$ dies unexpectedly sometime throughout
the protocol?

\begin{comment}
  At any time a task dies, the task info caps it held are released.
  Also, task death notifications are sent to any task that holds task
  info caps to the now dead task.  The task death notifications will
  be processed asynchrnouly, so they might be processed immediately,
  or at any later time, even much later after the task died!  So one
  important thing to keep in mind is that the release of task info
  caps a task held, and other tasks noticing the task death, are
  always some time apart.
\end{comment}

Because the client $C$ holds a task info cap for $S$ no imposter can
get the task ID of $S$.  $C$ and $D$ will get errors when trying to
send messages to $S$.

\begin{comment}
  You might now wonder what happens if $C$ also dies, or if $C$ is
  malicious and does not hold the task info cap.  You can use this as
  an exercise, and try to find the answer on your own.  The answers
  are below.
\end{comment}

Eventually, $C$ (and $D$ if it already got the task info cap for $S$)
will process the task death notification and clean up their state.

\subsubsection{The client $C$ dies}
The server $S$ and the destination task $D$ hold a task info cap for
$C$, so no imposter can get its task ID.  $S$ and $D$ will get errors
when trying to send messages to $C$.  Depending on when $C$ dies, the
capability might be copied successfully or not at all.

Eventually, $S$ and $D$ will process the task death notification and
release all resources associated with $C$.  If the reference was not
yet copied, this will include the reference container associated with
$C$, if any.  If the reference was already copied, this will only
include the empty reference container, if any.

\begin{comment}
  Of course, the participants need to use internal locking to protect
  the integrity of their internal data structures.  The above protocol
  does not show where locks are required.  In the few cases where some
  actions must be performed atomically, a wording is used that
  suggests that.
\end{comment}

\subsubsection{The destination task $D$ dies}

The client $C$ holds a task info cap for $D$ over the whole operation,
so no imposter can get its task ID.  Depending on when $D$ dies, it
has either not yet accepted the capability, then $C$ will clean up by
destroying the reference container, or it has, and then $S$ will clean
up its state when it processes the task death notification for $D$.

\subsubsection{The client $C$ and the destination task $D$ die}

This scenario is the reason why the server acquires its own task info
cap for $D$ so early, and why it must do that under the constraint
that $C$ still lives.  If $C$ and $D$ die before the server created
the reference container, then either no request was made, or creating
the task info cap for $D$ fails because of the constraint.  If $C$ and
$D$ die afterwards, then no imposter can get the task ID of $D$ and
try to get at the reference in the container, because the server has
its own task info cap for $D$.

\begin{comment}
  This problem was identified very late in the development of this
  protocol.  We just did not think of both clients dieing at the same
  time!  In an earlier version of the protocol, the server would
  acquire its task info cap when $D$ accepts its reference.  This is
  too late: If $C$ and $D$ die just before that, an imposter with
  $D$'s task ID can try to get the reference in the container before
  the server processes the task death notification for $C$ and
  destroys it.
\end{comment}

Eventually, the server will receive and process the task death
notifications.  If it processes the task death notification for $C$
first, it will destroy the whole container immediately, including the
reference, if any.  If it processes the task death notification for
$D$ first, it will destroy the reference, and leave behind the empty
container associated with $C$, until the other task death notification
is processed.  Either way no imposter can get at the capability.

Of course, if the capability was already copied at the time $C$ and
$D$ die, the server will just do the normal cleanup.

\subsubsection{The client $C$ and the server $S$ die}

This scenario does not cause any problems, because on the one hand,
the destination task $D$ holds a task info cap for $C$, and it
acquires its own task info cap for $S$.  Although it does this quite
late in the protocol, it does so under the constraint that $C$ still
lives, which has a task info cap for $S$ for the whole time (until it
dies).  It also gets the task info cap for $S$ before sending any
message to it.  An imposter with the task ID of $S$, which it was
possible to get because $C$ died early, would not receive any message
from $D$ because $D$ uses $C$ as its constraint in acquireing the task
info cap for $S$.

\subsubsection{The destination task $D$ and the server $S$ die}

As $C$ holds task info caps for $S$ and $D$, there is nothing that can
go wrong here.  Eventually, the task death notifications are
processed, but the task info caps are not released until the protocol
is completed or aborted because of errors.

\subsubsection{The client $C$, the destination task $D$ and the server $S$ die}

Before the last one of these dies, you are in one of the scenarios
which already have been covered.  After the last one dies, there is
nothing to take care of anymore.

\begin{comment}
  In this case your problem is probably not the capability copy
  protocol, but the stability of your software!  Go fix some bugs.
\end{comment}

So far the scenarios where one or more of the participating tasks die
unexpectedly.  They could also die purposefully.  Other things that
tasks can try to do purposefully to break the protocol are presented
in the following paragraphs.

\begin{comment}
  A task that tries to harm other tasks by not following a protocol
  and behaving as other tasks might expect it is malicious.  Beside
  security concerns, this is also an issue of robustness, because
  malicious behaviour can also be triggered by bugs rather than bad
  intentions.
  
  It is difficult to protect against malicious behaviour by trusted
  components, like the server $S$, which is trusted by both $C$ and
  $D$.  If a trusted component is compromised or buggy, ill
  consequences for software that trusts it must be expected.  Thus, no
  analysis is provided for scenarious involving a malicious or buggy
  server $S$.
\end{comment}

\subsubsection{The client $C$ is malicious}

If the client $C$ wants to break the protocol, it has numerous
possibilities to do so.  The first thing it can do is to provide a
wrong destination task ID when creating the container.  But in this
case, the server will return an error to $D$ when it tries to accept
it, and this will give $D$ a chance to notice the problem and clean
up.  This also would allow for some other task to receive the
container, but the client can give the capability to any other task it
wants to anyway, so this is not a problem.

\begin{comment}
  If a malicious behaviour results in an outcome that can also be
  achieved following the normal protocol with different parameters,
  then this not a problem at all.
\end{comment}

The client could also try to create a reference container for $D$ and
then not tell $D$ about it.  However, a reference container should not
consume a lot of resources in the server, and all such resources
should be attributed to $C$.  When $C$ dies eventually, the server
will clean up any such pending containers when the task death
notification is processed.

The same argument holds when $C$ leaves out the call to
\verb/cap_ref_cont_destroy/.

The client $C$ could also provide wrong information to $D$.  It could
supply a wrong server thread ID $T$.  It could supply a wrong
reference container ID $R$.  If $D$ does not trust $C$ and expects a
capability implemented by some specific trusted server, it will verify
the thread ID numerically and reject it if it does not match.  The
reference container ID will be verified by the server, and it will
only be accepted if the reference container was created by the client
task $C$.  Thus, the only wrong reference container IDs that the
client $C$ could use to not provoke an error message from the server
(which then lead $D$ to abort the operation) would be a reference
container that it created itself in the first place.  However, $C$
already is frree to send $D$ any reference container it created.

\begin{comment}
  Again $C$ can not achieve anything it could not achieve by just
  following the protocol as well.  If $C$ tries to use the same
  reference container with several RPCs in $D$, one of them would
  succeed and the others would fail, hurting only $C$.
  
  If $D$ does trust $C$, then it can not protect against malicious
  behaviour by $C$.
\end{comment}

To summarize the result so far: $C$ can provide wrong data in the
operations it does, but it can not achieve anything this way that it
could not achieve by just following the protocol.  In most cases the
operation would just fail.  If it leaves out some operations, trying
to provoke resource leaks in the server, it will only hurt itself (as
the reference container is strictly associated with $C$ until the
reference is accepted by $D$).

\begin{comment}
  For optimum performance, the server should be able to keep the
  information about the capabilities and reference containers a client
  holds on memory that is allocated on the clients behalf.
  
  It might also use some type of quota system.
\end{comment}

Another attack that $C$ can attempt is to deny a service that $S$ and
$D$ are expecting of it.  Beside not doing one or more of the RPCs,
this is in particular holding the task info caps for the time span as
described in the protocol.  Of course, this can only be potentially
dangerous in combination with a task death.  If $C$ does not hold the
server task info capability, then an imposter of $S$ could trick $D$
into using the imposter as the server.  However, this is only possible
if $D$ already trusts $C$.  Otherwise it would only allow servers that
it already trusts, and it would always hold task info caps to such
trusted servers when making the decision that it trusts them.
However, if $D$ trusts $C$, it can not protect against $C$ being
malicious.

\begin{comment}
  If $D$ does not trust $C$, it should only ever compare the task ID
  of the server thread against trusted servers it has a task info cap
  for.  It must not rely on $C$ doing that for $D$.
  
  However, if $D$ does trust $C$, it can rely on $C$ holding the
  server task info cap until it got its own.  Thus, the task ID of $C$
  can be used as the constraint when acquiring the task info cap in
  the protocol.
\end{comment}

If $C$ does not hold the task info cap of $D$, and $D$ dies before the
server acquires its task info cap for $D$, it might get a task info
cap for an imposter of $D$.  But if the client wants to achieve that,
it could just follow the protocol with the imposter as the destination
task.

\subsubsection{The destination task $D$ is malicious}

The destination task has not as many possibilities as $C$ to attack
the protocol.  This is because it is trusted by $C$.  So the only
participant that $D$ can try to attack is the server $S$.  But the
server $S$ does not rely on any action by $D$.  $D$ does not hold any
task info caps for $S$.  The only operation it does is an RPC to $S$
accepting the capability, and if it omits that it will just not get
the capability (the reference will be cleaned up by $C$ or by the
server when $C$ dies).

The only thing that $D$ could try is to provide false information in
the \verb/cap_ref_cont_accept/ RPC.  The information in that RPC is
the task ID of the client $C$ and the reference container ID $R$.  The
server will verify that the client $C$ has previously created a
reference container with the ID $R$ that is destined for $D$.  So $D$
will only be able to accept references that it is granted access to.
So it can not achieve anything that it could not achieve by following
the protocol (possibly the protocol with another client).  If $D$
accepts capabilities from other transactions outside of the protocol,
it can only cause other transactions in its own task to fail.

\begin{comment}
  If you can do something wrong and harm yourself that way, then this
  is called ``shooting yourself in your foot''.
  
  The destination task $D$ is welcome to shoot itself in its foot.
\end{comment}

\subsubsection{The client $C$ and the destination task $D$ are malicious}

The final question we want to raise is what can happen if the client
$C$ and the destination task $D$ are malicious.  Can $C$ and $D$
cooperate and attacking $S$ in a way that $C$ or $D$ alone could not?

In the above analysis, there is no place where we assume any specific
behaviour of $D$ to help $S$ in preventing an attack on $S$.  There is
only one place where we make an assumption for $C$ in the analysis of
a malicious $D$.  If $D$ does not accept a reference container, we
said that $C$ would clean it up by calling
\verb/cap_ref_cont_destroy/.  So we have to look at what would happen
if $C$ were not to do that.

Luckily, we covered this case already.  It is identical to the case
where $C$ does not even tell $D$ about the reference container and
just do nothing.  In this case, as said before, the server will
eventually release the reference container when $C$ dies.  Before
that, it only occupies resources in the server that are associated
with $C$.

This analysis is sketchy in parts, but it covers a broad range of
possible attacks.  For example, all possible and relevant combinations
of task deaths and malicious tasks are covered.  Although by no means
complete, it can give us some confidence about the rightness of the
protocol.  It also provides a good set of test cases that you can test
your own protocols, and improvements to the above protocol against.


\subsection{The trust rule}

The protocol to copy a capability from one client to another task has
a dramatic consequence on the design of the Hurd interfaces.

Because the receiver of the capability must make blocking calls to the
server providing the capability, the receiver of the capability
\emph{must} trust the server providing the capability.

This means also: If the receiver of a capability does not trust the
server providing the capability, it \emph{must not} accept it.

The consequence is that normally, servers can not accept capabilities
from clients, unless they are provided by a specific trusted server.
This can be the \texttt{task} or \texttt{auth} server for example.

This rule is even true if the receiver does not actually want to use
the capability for anything.  Just accepting the capability requires
trusting the server providing it already.

In the Hurd on Mach, ports (which are analogous to capabilities in
this context) can be passed around freely.  There is no security risk
in accepting a port from any source, because the kernel implements
them as protected objects.  Using a port by sending blocking messages
to it requires trust, but simply storing the port on the server side
does not.

This is different in the Hurd on L4: A server must not accept
capabilities unless it trusts the server providing them.  Because
capabilities are used for many different purposes (remote objects,
authentication, identification), one has to be very careful in
designing the interfaces.  The Hurd interfaces on Mach use ports in a
way that is not possible on L4.  Such interfaces need to be
redesigned.

Often, redesigning such an interface also fixes some other security
problems that exists with in the Hurd on L4, in particular DoS
attacks.  A good part of this paper is about redesigning the Hurd to
avoid storing untrusted capabilities on the server side.

\begin{comment}
  Examples are:

  \begin{itemize}
  \item The new authentication protocol, which eliminates the need for
    a rendezvous port and is not only faster, but also does not
    require the server to block on the client anymore (see section
    \ref{auth} on page \pageref{auth}).
    
  \item The signal handling, which does not require the \texttt{proc}
    server to hold the signal port for every task anymore (see section
    \ref{signals} on page \pageref{signals}).
    
  \item The new exec protocol, which eliminates the need to pass all
    capabilities that need to be transfered to the new executable from
    the old program to the filesystem server, and then to the
    \texttt{exec} server (see section \ref{exec} on page
    \pageref{exec}).
    
  \item The new way to implement Unix Domain Sockets, which don't
    require a trusted system server, so that descriptor passing (which
    is really capability passing) can work (see section
    \ref{unixdomainsockets} on page \pageref{unixdomainsockets}.

  \item The way parent and child filesystem are linked to each other,
    in other words: how mounting a filesystem works (see section
    \ref{xfslookup} on page \pageref{xfslookup}).
    
  \item The replacement for the \verb/file_reparent()/ RPC (see
    section \ref{reparenting} on page \pageref{reparenting}).
  \end{itemize}
\end{comment}

\section{Synchronous IPC}

The Hurd only needs synchronous IPC.  Asynchronous IPC is usually not
required.  An exception are notifications (see below).

There are possibly some places in the Hurd source code where
asynchronous IPC is assumed.  These must be replaced with different
strategies.  One example is the implementation of select() in the GNU
C library.

\begin{comment}
  A naive implementation would use one thread per capability to select
  on.  A better one would combine all capabilities implemented by the
  same server in one array and use one thread per server.
  
  A more complex scheme might let the server process select() calls
  asynchronously and report the result back via notifications.
\end{comment}

In other cases the Hurd receives the reply asynchronously from sending
the message.  This works fine in Mach, because send-once rights are
used as reply ports and Mach guarantees to deliver the reply message,
ignoring the kernel queue limit.  In L4, no messages are queued and
such places need to be rewritten in a different way (for example using
extra threads).

\begin{comment}
  What happens if a client does not go into the receive phase after a
  send, but instead does another send, and another one, quickly many
  sends, as fast as possible?  A carelessly written server might
  create worker threads for each request.  Instead, the server should
  probably reject to accept a request from a client thread that
  already has a pending request, so the number of worker threads is
  limited to the number of client threads.
  
  This also makes interrupting an RPC operation easier (the client
  thread ID can be used to identify the request to interrupt).
\end{comment}


\section{Notifications}

Notifications to untrusted tasks happen frequently.  One case is
object death notifications, in particular task death notifications.
Other cases might be select() or notifications of changes to the
filesystem.

The console uses notifications to broadcast change events to the
console content, but it also uses shared memory to broadcast the
actual data, so not all notifications need to be received for
functional operation.  Still, at least one notification is queued by
Mach, and this is sufficient for the console to wakeup whenever
changes happened, even if the changes can not be processed
immediately.
  
From the servers point of view, notifications are simply messages with
a send and xfer timeout of 0 and without a receive phase.

For the client, however, there is only one way to ensure that it will
receive the notification: It must have the receiving thread in the
receive phase of an IPC.  While this thread is processing the
notification (even if it is only delegating it), it might be preempted
and another (or the same) server might try to send a second
notification.

\begin{comment}
  It is an open challenge how the client can ensure that it either
  receives the notification or at least knows that it missed it, while
  the server remains save from potential DoS attacks.  The usual
  strategy, to give receivers of notifications a higher scheduling
  priority than the sender, is not usable in a system with untrusted
  receivers (like the Hurd).  The best strategy determined so far is
  to have the servers retry to send the notification several times
  with small delays inbetween.  This can increase the chance that a
  client is able to receive the notification.  However, there is still
  the question what a server can do if the client is not ready.
 
  An alternative might be a global trusted notification server that
  runs at a higher scheduling priority and records which servers have
  notifications for which clients, and that can be used by clients to
  be notified of pending notifications.  Then the clients can poll the
  notifications from the servers.
\end{comment}


