\chapter{Virtual Memory Management}

\begin{quote}
\emph{The mind and memory are more sharply exercised in comprehending
another man's things than our own.}

\begin{flushright}
\emph{Timber} or \emph{Discoveries} by Ben Jonson
\end{flushright}
\end{quote}


\section{Introduction}

The goal of an operating system is simply, perhaps reductively,
stated: manage the available resources.  In other words, it is the
operating system's job to dictate the policy for obtaining resources
and to provide mechanisms to use them.  Most resources which the
operating system manages are sparse resources, for instance the CPUs,
the memory and the various peripherals including graphics cards and
hard drives.  Any given process, therefore, needs to compete with the
other processes in the system for some subset of the available
resources at any given time.  As can be imagined, the policy to access
and the mechanisms to use these resources determines many important
characteristics of the system.

A simple single user system may use a trivial first come first serve
policy for allocating resources, a device abstraction layer and no
protection domains.  Although this design may be very light-weight and
the thin access layer conducive to high speed, this design will only
work on a system where all programs can be trusted: a single malicious
or buggy program can potentially halt all others from making progress
simply by refusing to yield the CPU or allocating and not releasing
resources in a timely fashion.

The Hurd, like Unix, aims to provide strong protection domains thereby
preventing processes from accidentally or maliciously harming the rest
of the system.  Unix has shown that this can be done efficiently.  But
more than Unix, the Hurd desires to identify pieces of the system
which Unix placed in the kernel but which need not be there as they
could be done in user space and provide additional user flexibility.
Through our experience and analysis, we are convinced that one area is
much of the virtual memory system: tasks are often allocating as much
memory without regard---because Unix provides them with no mechanism
to do so---for the rest of the system.  But it is not a cooperative
model which we wish to embrace but a model which holds the users of
the resource responsible for it and when asked to release some of its
memory will or violate the social contract and face exile.  Not only
will this empower users but it will force them to make smarter
decisions.

\subsection{Learning from Unix}

Unix was designed as a multiuser timesharing system with protection
domains thereby permitting process separation, i.e. allowing different
users to concurrently run processes in the system and gain access to
resources in a controlled fashion such that any one process cannot
hurt or excessively starve any other.  Unix achieved this through a
monolithic kernel design wherein both policy and mechanism are
provided by the kernel.  Due to the limited hardware available at the
time and the state of Multics\footnote{Multics was seen as a system
which would never realize due to its overly ambitious feature set.},
Unix imposed a strong policy on how resources could be used: a program
could access files, however, lower level mechanism such as the file
system, the virtual file system, network protocol stacks and devices
drivers all existed in the kernel proper.  This approach made sense
for the extremely limited hardware that Unix was targeted for in the
1970s.  As hardware performance increased, however, a separation
between mechanism and policy never took place and today Unix-like
operating systems are in a very similar state to those available two
decades ago; certainly, the implementations have been vastly improved
and tuned, however, the fundamental design remains the same.

One of the most important of the policy/mechanism couplings in the
kernel is the virtual memory subsystem: every component in the system
needs memory for a variety of reasons and with different priorities.
The system must attempt to meet a given allocation criteria.  However,
as the kernel does not and cannot know how how a task will use its
memory except based on the use of page fault statistics is bound to
make sub-ideal eviction decisions.  It is in part through years of
fine tuning that Unix is able to perform as well as it does for the
general applications which fit its assumed statistical model.

\subsection{Learning from Mach}

The faults of Unix became clear through the use of Mach.  The
designers of Mach observed that there was too much mechanism in the
kernel and attempted to export the file systems, network stack and
much of the system API into user space servers.  They left a very
powerful VMM in the kernel with the device drivers and a novel IPC
system.  Our experience shows that the VMM although very flexible, is
unable to make smart paging decisions: because Unix was tied to so
many subsystems, it had a fair knowledge of how a lot of the memory in
the system was being used.  It could therefore make good guesses about
what memory could be evicted and not be needed in the near future.
Mach, however, did not have this advantage and relied strictly on page
fault statistics and access pattern detection for its page eviction
policy.

Based on this observation, it is imperitive that the page eviction
scheme have good knowledge about how pages are being used as it only
requires a few bad decisions to destroy performance.  Thus, a new
design can either choose to return to the monolithic design and add
even more knowledge to the kernel to increase performance or the page
eviction scheme can be remove from the kernel completely and placed in
user space and make all tasks self paged.

\subsection{Following the Hurd Philosophy}

As the Hurd aims, like Unix, to be a multiuser system for mutually
untrusted users, security is an absolute necessity.  But it is not the
object of the system to limit users excessively: as long as operations
can be done securely, they should be permitted.  It is based on this
philosophy that we have adopted a self paging design for the new Hurd
VMM: who knows better how a task will use its memory than the task
itself?  This is clear from the problems that have been encountered
with LRU, the basic page evition algorithm, by database developers,
language designers implementing garbage collectors and soft realtime
application developers such as multimedia developers: they all wrestle
with the underlying operating system's page eviction scheme.  By
putting the responsibility to page on tasks we think that tasks will
be forced to make smart decisions as they can only hurt themselves.

\section{Memory Allocation}

If memory was infinite and the only problem was worrying about one
program accessing the memory of another, memory allocation would be
trivial.  This is not, however, the case: memory is visibly finite and
a well designed system will exploit it all.  As memory is a system
resource, a system wide memory allocation policy must be established
which maximizes memory usage according to a given set of criteria.

In a typical Unix-like VMM, allocating memory (e.g. using
\function{sbrk} or \function{mmap}) does not allocate physical memory
but \keyword{virtual memory}.  In order to increase the amount of
memory available to users, the kernel uses a \keyword{backing store},
typically a hard disk, to temporarily free physical memory thereby
allowing other processes to make progress.  The sum of these two is
referred to as virtual memory.  The use of backing store ensures data
integrity when physical memory must be freed and application
transparency is required.  A variety of criteria are used to determine
which frames are \keyword{paged out}, however, most often some form of
a priority based least recently used, LRU, algorithm is applied.  Upon
\keyword{memory pressure}, the system steals pages from low priority
processes which have not been used recently or drain pages from an
internal cache.

This design has a major problem: the kernel has to evict the pages but
only the applications know which pages they really need in the near
term.  The kernel could ask the applications for this data, however,
it is unable to trust the applications as they could, for instance,
not respond, and the kernel would have to forcefully evict pages
anyway.  As such, the kernel relies on page fault statistics to make
projections about how the memory will be used, thus the LRU eviction
scheme.  An additional result of this scheme is that as applications
never know if mapped memory is in core, they are unable to make
guarantees about deadlines.

These problems are grounded in the way the Unix VMM allocates memory:
it does not allocate physical memory but virtual memory.  This is
illustated by the following scenario: when a process starts and begins
to use memory, the allocator will happily give it all of memory in the
system as long as no other process wants it.  What happens, however,
when a second memory hungry process starts is that the kernel has no
way to take back memory it allocated to the first process.  At this
point, it has two options: it can either return failure to the second
process or it can steal memory from the first process and send it to
backing store.

One way to solve these problems is to have the VMM allocate phsyical
memory and make applications completely self-paged.  Thus, the burden
of paging lies the application themselves.  When application request
memory, they no longer request virutal memory but physical memory.
Once the application has exhausted its available frames, it is its
responsibility to multiplex the available frames.  Thus, virtual
memory is done in the application itself.  It is important to note
that a standard manager or managers should be supplied by the
operating system.  This is important for implementing something like a
POSIX personality.  This should not, however, be hard coded: certain
application may greatly benefit by being able to control their own
eviction schemes.  At its most basic level, hints could be provided to
the manager by introducing extentions on basic function calls.  For
instance, \function{malloc} could take an extra parameter indicating
the class of data being allocated.  These class would provide hints
about the expected usage pattern and life time of the data.

\subsection{Bootstrap}

When the Hurd starts up, all physical memory is eventually transfered
to the physical memory server by the root server.  At this point, the
physical memory server will control all of the physical pages in the
system.

\subsection{Allocation Policy}

The physical memory server maintains a concept of \keyword{guaranteed
pages} and \keyword{extra pages}.  The former are pages that a given
task is guaranteed to map in a very short amount of time.  Given this
predicate, the total number of guaranteed pages can never exceed the
total number of frames in the system.  Extra pages are pages which are
given to clients who have reached their guaranteed page allocation
limit.  The phsyical memory server may request that a client
relinquish a number of extant extra pages at any time.  The client
must return the pages to the physical memory (i.e. free them) in a
short amount of time.  Should a task fail to do this, it risks having
all of its memory dropped (i.e. not swapped out or saved in anyway)
and reclaimed by the physical memory server.

Readers familiar with VMS will see a striking difference between these
two systems.  This is not without reason.  Yet, differences remains:
VMS does not have extra pages and the number of pages is fixed at task
creation time.  VMS than maintains a dirty list of pages thereby
having a very fast backing store and essentially allowing tasks to
have more than their quota of memory if there is no memory pressure.
One reason that this is copied in this design is that unlike in VMS,
the file systems and device drivers are in user space.  Thus, the
caching that was being done by VMS can not be done intelligently by
the physical memory server.

The number of guaranteed pages that a given task has access to is not
determined by the physical memory server but by the \keyword{memory
policy server}.  This division allows the physical memory server to
only concern itself with the mechanisms and means that it must know
essentially nothing about how the underlying operating system
functions.  (The implication is that although tailored for Hurd
specific needs, the physical memory server is completely separate from
the Hurd and can be used by other operating systems running on the
microkernel.)  Thus, it is the memory policy server's responsibility
to determine who gets how much memory.  This may be determined as a
function of the user or looking in file on disk for e.g. quotas.  As
can be seen this type of data acquisition could add significant
complexity to the physical memory server and require blocking states
(e.g. waiting for a read operation on file i/o) and could create
circular dependencies.

The physical memory server and the memory policy server will contain a
shared buffer of tupples indexed by task id containing the number of
allocated pages, the number of guaranteed page, and a boolean
indicating whether or not this task is eligible for guaranteed pages.
The guaranteed page field and the extra page predicate may only be
written to by the memory policy server.  The number of allocated pages
may only be written to by the physical memory server.  This scheme
means that no locking in required.  (On some architectures where a
read of a given field cannot be performed in a single operation, the
read may have to be done twice).

Until the memory policy server makes the intial contact with the
physical memory server, memory will be allocated on a first come first
serve basis.  The memory policy server shall use the following remote
procedure call to contact the physical memory server:

\begin{code}
error\_t physical\_memory\_server\_introduce (void)
\end{code}

\noindent
This function will succeed the first time it is called.  It will fail
all subsequent times.  The physical memory server will record the
sender of this rpc as the memory policy server and begin allocating
memory according to the previously described protocol.

The shared policy buffer may be obtained from the physical memory
server by the policy by calling:

\begin{code}
error\_t physical\_memory\_server\_get\_policy\_buffer (out l4\_map\_t buffer)
\end{code}

\noindent
The returned buffer is mapped with read and write access into the
policy memory server's address space.  It may need to be resized.  If
this is the case, the physical memory server shall unmap the buffer
from the policy memory server's address space, copy the buffer
internally as required.  The policy memory server will fault on the
memory region on its next access and it may repeat the call.  This
call will succeed when the sender is the memory policy server, it will
fail otherwise.

\subsection{Allocation Mechanisms}

Applications are able allocate memory by  Memory allocation will be 


% Traditionally, monolithical kernels, but even kernels like Mach,
% provide a virtual memory management system in the kernel.  All paging
% decisions are made by the kernel itself.  This requires good
% heuristics.  Smart paging decisions are often not possible because the
% kernel lacks the information about how the data is used.
% 
% In the Hurd, paging will be done locally in each task.  A physical
% memory server provides a number of guaranteed physical pages to tasks.
% It will also provide a number of excess pages (over-commit).  The task
% might have to return any number of excess pages on short notice.  If
% the task does not comply, all mappings are revoked (essentially
% killing the task).
% 
% A problem arises when data has to be exchanged between a client and a
% server, and the server wants to have control over the content of the
% pages (for example, pass it on to other servers, like device drivers).
% The client can not map the pages directly into the servers address
% space, as it is not trusted.  Container objects created in the
% physical memory server and mapped into the client and/or the servers
% address space will provide the necessary security features to allow
% this.  This can be used for DMA and zero-copying in the data exchange
% between device drivers and (untrusted) user tasks.
% 
% 
